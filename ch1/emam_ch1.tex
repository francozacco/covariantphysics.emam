\documentclass[11pt]{article}
\usepackage{amssymb}
\usepackage{amsthm}
\usepackage{enumitem}
\usepackage{amsmath}
\usepackage{bm}
\usepackage{adjustbox}
\usepackage{mathrsfs}
\usepackage{graphicx}
\usepackage{siunitx}
\usepackage{physics}
\usepackage[mathscr]{euscript}

\title{\textbf{Solved selected problems of Covariant Physics by Moataz Emam}}
\author{Franco Zacco}
\date{}

\addtolength{\topmargin}{-3cm}
\addtolength{\textheight}{3cm}

\newcommand{\R}{\mathbb{R}}
\newcommand{\C}{\mathbb{C}}
\newcommand{\Z}{\mathbb{Z}}
\newcommand{\hatr}{\bm{\hat{r}}}
\newcommand{\hatx}{\bm{\hat{x}}}
\newcommand{\haty}{\bm{\hat{y}}}
\newcommand{\hatz}{\bm{\hat{z}}}
\newcommand{\hatth}{\bm{\hat{\theta}}}
\newcommand{\hatphi}{\bm{\hat{\phi}}}
\newcommand{\hatrho}{\bm{\hat{\rho}}}
\newcommand{\ei}[1]{\hat{\bm{e}}_#1}
\newcommand{\Lagr}{\mathcal{L}}
\theoremstyle{definition}
\newtheorem*{solution*}{Solution}
\renewcommand*{\proofname}{\textbf{Solution}}

\begin{document}
\maketitle
\thispagestyle{empty}

\begin{proof}{\textbf{Exercise 1.1}}
\begin{itemize}
\item [2.] For the oblate spheroidal coordinates system if we fix $\mu$ to
different values we get a set of concentric spheres as we show below
% \begin{center}
%     \includegraphics[scale=0.3]{ch1_1.1-i.png}
% \end{center}
If we fix $\nu$ we get the following surfaces
% \begin{center}
%     \includegraphics[scale=0.3]{ch1_1.1-ii.png}
% \end{center}
Finally if we fix $\phi$ to different values we get that
% \begin{center}
%     \includegraphics[scale=0.3]{ch1_1.1-iii.png}
% \end{center}
Now, joining all together we get that
% \begin{center}
%     \includegraphics[scale=0.3]{ch1_1.1-iv.png}
% \end{center}
\end{itemize}
\end{proof}

\cleardoublepage
\begin{proof}{\textbf{Exercise 1.2}}\\
Let us consider a point in the $x', y', z'$ coordinate system as shown below
% \begin{center}
%     \includegraphics[scale=0.35]{ch1_1.2.png}
% \end{center}
Then we see that
\begin{align*}
    x &= x'\\
    y &= y' + z'\sin\alpha\\
    z &= z'\cos\alpha
\end{align*}
Or
\begin{align*}
    x' &= x\\
    y' &= y - z \tan\alpha\\
    z' &= \frac{z}{\cos\alpha}
\end{align*}
\end{proof}

\cleardoublepage
\begin{proof}{\textbf{Exercise 1.3}}\\
\begin{itemize}
\item [1.] We know that in spherical coordinates
\begin{align*}
    x^1 &= r\sin\theta\cos\phi\\
    x^2 &= r\sin\theta\sin\phi\\
    x^3 &= r\cos\theta
\end{align*}
Then
\begin{align*}
    dx^1 &= d(r\sin\theta\cos\phi)\\
    &= dr\sin\theta\cos\phi + rd(\sin\theta)\cos\phi + r\sin\theta d(\cos\phi)\\
    &= dr\sin\theta\cos\phi + rd\theta\cos\theta\cos\phi - rd\phi\sin\theta \sin\phi
    \\\\
    dx^2 &= d(r\sin\theta\sin\phi)\\
    &= dr\sin\theta\sin\phi + rd(\sin\theta)\sin\phi + r\sin\theta d(\sin\phi)\\
    &= dr\sin\theta\sin\phi + rd\theta\cos\theta\sin\phi + rd\phi\sin\theta \cos\phi
    \\\\
    dx^3 &= d(r\cos\theta)\\
    &= dr\cos\theta + rd(\cos\theta)\\
    &= dr\cos\theta - rd\theta\sin\theta
\end{align*}
Let us compute each component $(dx^i)^2$ separately
\begin{align*}
    (dx^1)^2 &= (dr\sin\theta\cos\phi + rd\theta\cos\theta\cos\phi - rd\phi\sin\theta \sin\phi)^2\\
    &= dr^2\sin^2\theta\cos^2\phi + r^2d\theta^2\cos^2\theta\cos^2\phi
    + r^2d\phi^2\sin^2\theta \sin^2\phi\\
    &\quad+ 2rdr d\theta\sin\theta\cos\theta\cos^2\phi
    - 2rdrd\phi\sin^2\theta\cos\phi\sin\phi\\
    &\quad- 2r^2d\theta d\phi\cos\theta\cos\phi \sin\theta \sin\phi
    \\\\
    (dx^2)^2 &= (dr\sin\theta\sin\phi + rd\theta\cos\theta\sin\phi + rd\phi\sin\theta \cos\phi)^2\\
    &= dr^2\sin^2\theta\sin^2\phi + r^2d\theta^2\cos^2\theta\sin^2\phi + r^2d\phi^2\sin^2\theta\cos^2\phi\\
    &\quad+ 2rdrd\theta\sin\theta\cos\theta\sin^2\phi
    + r^2d\theta d\phi\cos\theta\sin\phi\sin\theta \cos\phi\\
    &\quad+ rdrd\phi\sin^2\theta\sin\phi\cos\phi
    \\\\
    (dx^3)^2 &= (dr\cos\theta - rd\theta\sin\theta)^2\\
    &= dr^2\cos^2\theta + r^2d\theta^2\sin^2\theta
    - 2rdrd\theta\cos\theta\sin\theta
\end{align*}
Now we sum all components
\begin{align*}
    dl^2 &= (dx^1)^2 + (dx^2)^2 + (dx^3)^2
    = dr^2 + r^2d\theta^2 + r^2\sin^2\theta d\phi^2
\end{align*}
Where we see that all terms with two different differentials cancel out.

\item[2.] We know that the spherical coordinates relate to the cylindrical
coordinates as follows
\begin{align*}
    \rho &= r\sin\theta\\
    z &= r\cos\theta\\
    \phi &= \phi
\end{align*}
Then
\begin{align*}
    d\rho &= dr\sin\theta + rd\theta \cos\theta\\
    dz &= dr\cos\theta - rd\theta \sin\theta\\
    d\phi &= d\phi
\end{align*}
Also, we know that the line element in cylindrical coordinates is given by
\begin{align*}
    dl^2 = d\rho^2 + \rho^2d\phi^2 + dz^2
\end{align*}
Hence replacing we get that
\begin{align*}
    dl^2 &= (dr\sin\theta + rd\theta \cos\theta)^2
    + r^2\sin^2\theta d\phi^2 + (dr\cos\theta - rd\theta \sin\theta)^2\\
    &= dr^2\sin^2\theta + r^2d\theta^2 \cos^2\theta
    + 2rdrd\theta\sin\theta\cos\theta\\
    &\quad+ r^2\sin^2\theta d\phi^2 + dr^2\cos^2\theta + r^2d\theta^2 \sin^2\theta
    -2rdrd\theta\cos\theta\sin\theta\\
    &= dr^2 + r^2d\theta^2 + r^2\sin^2\theta d\phi^2
\end{align*}

\item[5.] We know that in the skew coordinate system we have
\begin{align*}
    x &= x'\\
    y &= y' +  z'\sin\alpha\\
    z &= z'\cos\alpha
\end{align*}
Then
\begin{align*}
    dx &= dx'\\
    dy &= dy' +  dz'\sin\alpha\\
    dz &= dz'\cos\alpha
\end{align*}
Hence the line element in the skew coordinate system is
\begin{align*}
    dl^2 &= dx^2 + dy^2 + dz^2\\
    &= dx'^2 + (dy' + dz'\sin\alpha)^2 + \cos^2\alpha dz'^2\\
    &= dx'^2 + dy'^2 + \sin^2\alpha dz'^2 + 2dy'dz'\sin\alpha + \cos^2\alpha dz'^2\\
    &= dx'^2 + dy'^2 + dz'^2 + 2dy'dz'\sin\alpha
\end{align*}
We see that the metric is not diagonal as expected.
\end{itemize}
\end{proof}

\cleardoublepage
\begin{proof}{\textbf{Exercise 1.4}}\\
\begin{itemize}
\item [3.] The Oblate spheroidal coordinates are defined by
\begin{align*}
    x &= a\cosh\mu\cos\nu\cos\varphi\\
    y &= a\cosh\mu\cos\nu\sin\varphi\\
    z &= a\sinh\mu\sin\nu
\end{align*}
Then the scale factors are
\begin{align*}
    h_\mu^2
    &= a^2\sinh^2\mu\cos^2\nu\cos^2\varphi + a^2\sinh^2\mu\cos^2\nu\sin^2\varphi
    + a^2\cosh^2\mu\sin^2\nu\\
    &= a^2\sinh^2\mu\cos^2\nu(\cos^2\varphi + \sin^2\varphi)
    + a^2\cosh^2\mu\sin^2\nu\\
    &= a^2\sinh^2\mu\cos^2\nu + a^2\cosh^2\mu\sin^2\nu\\
    &= a^2(\sinh^2\mu\cos^2\nu + (\sinh^2\mu + 1)\sin^2\nu)\\
    &= a^2(\sinh^2\mu + \sin^2\nu)
    \\\\
    h_\nu^2
    &= a^2\cosh^2\mu\sin^2\nu\cos^2\varphi + a^2\cosh^2\mu\sin^2\nu\sin^2\varphi
    + a^2\sinh^2\mu\cos^2\nu\\
    &= a^2\cosh^2\mu\sin^2\nu + a^2\sinh^2\mu\cos^2\nu\\
    &= a^2(\sinh^2\mu + \sin^2\nu)
    \\\\
    h_\varphi^2
    &= a^2\cosh^2\mu\cos^2\nu\sin^2\varphi + a^2\cosh^2\mu\cos^2\nu\cos^2\varphi + 0\\
    &= a^2\cosh^2\mu\cos^2\nu
\end{align*}
Hence the line element is given by
\begin{align*}
    dl^2 &= a^2(\sinh^2\mu + \sin^2\nu)d\mu^2
    + a^2(\sinh^2\mu + \sin^2\nu)d\nu^2
    + a^2\cosh^2\mu\cos^2\nu d\varphi^2\\
    &= a^2(\sinh^2\mu + \sin^2\nu)(d\mu^2 + d\nu^2)
    + a^2\cosh^2\mu\cos^2\nu d\varphi^2\\
\end{align*}
\end{itemize}
\end{proof}

\cleardoublepage
\begin{proof}{\textbf{Exercise 1.6}}\\
Equation (1.57) states that
\begin{align*}
    dl^2 = \delta_{ij}dx^i dx^j
\end{align*}
If we expand it we get that
\begin{align*}
    dl^2 &= \delta_{11}dx^1 dx^1 + \delta_{12}dx^1 dx^2 + \delta_{13}dx^1 dx^3
    + \delta_{21}dx^2 dx^1 + \delta_{22}dx^2 dx^2\\
    &\quad+ \delta_{23}dx^2 dx^3
    + \delta_{31}dx^3 dx^1 + \delta_{32}dx^3 dx^2 + \delta_{33}dx^3 dx^3\\
    &= \delta_{11}(dx^1)^2 + \delta_{12}dx^1 dx^2 + \delta_{13}dx^1 dx^3
    + \delta_{21}dx^2 dx^1 + \delta_{22}(dx^2)^2\\
    &\quad+ \delta_{23}dx^2 dx^3
    + \delta_{31}dx^3 dx^1 + \delta_{32}dx^3 dx^2 + \delta_{33}(dx^3)^2
\end{align*}
but we know that $\delta_{ij} = 1$ if $i = j$ and $\delta_{ij} = 0$ if
$i \neq j$ then
\begin{align*}
    dl^2 &= (dx^1)^2 + (dx^2)^2 + (dx^3)^2
\end{align*}
Which is equation (1.15).
\end{proof}

\cleardoublepage
\begin{proof}{\textbf{Exercise 1.7}}\\
\begin{itemize}
\item [1.]
\begin{align*}
    (\bm{A}\cdot\bm{B})(\bm{C}\cdot\bm{D})
    &= (A^iB^j \delta_{ij})(C^kD^l \delta_{kl})
    = (A^iB_i)(C^kD_k)
\end{align*}
\item [2.]
\begin{align*}
    (\bm{A} - \bm{B})\cdot\hat{\bm{n}} &= 0 \\
    (A^i - B^i)n^j\delta_{ij} &= 0\\
    (A^i - B^i)n_i &= 0
\end{align*}
The condition for $\hat{\bm{n}}$ to have unit magnitude is
$n^in^j\delta_{ij} = 1$.
\item [3.]
\begin{align*}
    \cos\theta &= \frac{\bm{A} \cdot \bm{B}}{AB}
    = \frac{A^iB^j\delta_{ij}}{\sqrt{A^iA^j\delta_{ij}}\sqrt{B^kB^l\delta_{kl}}}
\end{align*}
\item [4.]
\begin{align*}
    \bm{E} = k\frac{\bm{r}}{r^3}
    = k\frac{x^i\hat{\bm{e}}_i}{(\sqrt{x^ix^j\delta_{ij}})^3}
    = k\frac{x^i\hat{\bm{e}}_i}{(x^ix^j\delta_{ij})^{3/2}}
\end{align*}
\item [5.]
\begin{align*}
    dW = \bm{F}\cdot d\bm{r} = F^i dx^j \delta_{ij}
\end{align*}
\item [6.]
\begin{align*}
    \frac{1}{2}mv^2 = \frac{1}{2}m(v^iv^j\delta_{ij})
\end{align*}
\end{itemize}
\end{proof}

\cleardoublepage
\begin{proof}{\textbf{Exercise 1.8}}\\
\begin{itemize}
\item [1.]
\begin{align*}
    A^1_3B_1^2 + A^2_3B_2^2 + A^3_3B_3^2 = A^i_3B_i^2
\end{align*}
\item [2.]
\begin{align*}
    A^1_{11} + A^2_{12} + A^3_{13} = A^i_{1i}
\end{align*}
\item [3.] The three expressions
\begin{align*}
    A^{11}B_1 + A^{12}B_2 + A^{13}B_3 &= C^1\\
    A^{21}B_1 + A^{22}B_2 + A^{23}B_3 &= C^2\\
    A^{31}B_1 + A^{32}B_2 + A^{33}B_3 &= C^3
\end{align*}
can be reduced to 
\begin{align*}
    A^{ji}B_i &= C^j
\end{align*}
\end{itemize}
\end{proof}

\cleardoublepage
\begin{proof}{\textbf{Exercise 1.9}}\\
\begin{itemize}
\item [1.] Assuming each vector has 3 components we get that
\begin{align*}
    A_i^jB_jC^i &= A_1^1B_1C^1 + A_2^1B_1C^2 + A_3^1B_1C^3\\
    &\quad+ A_1^2B_2C^1 + A_2^2B_2C^2 + A_3^2B_2C^3\\
    &\quad+ A_1^3B_3C^1 + A_2^3B_3C^2 + A_3^3B_3C^3
\end{align*}
\item [2.] Assuming each vector has 3 components we get that
\begin{align*}
    A_iB^j_kC^kD_j^i &= A_1B^1_1C^1D_1^1 + A_2B^1_1C^1D_1^2 + A_3B^1_1C^1D_1^3\\
    &\quad+ A_1B^2_1C^1D_2^1 + A_2B^2_1C^1D_2^2 + A_3B^2_1C^1D_2^3\\
    &\quad+ A_1B^3_1C^1D_3^1 + A_2B^3_1C^1D_3^2 + A_3B^3_1C^1D_3^3\\
    %
    &\quad+ A_1B^1_2C^2D_1^1 + A_2B^1_2C^2D_1^2 + A_3B^1_2C^2D_1^3\\
    &\quad+ A_1B^2_2C^2D_2^1 + A_2B^2_2C^2D_2^2 + A_3B^2_2C^2D_2^3\\
    &\quad+ A_1B^3_2C^2D_3^1 + A_2B^3_2C^2D_3^2 + A_3B^3_2C^2D_3^3\\
    %
    &\quad+ A_1B^1_3C^3D_1^1 + A_2B^1_3C^3D_1^2 + A_3B^1_3C^3D_1^3\\
    &\quad+ A_1B^2_3C^3D_2^1 + A_2B^2_3C^3D_2^2 + A_3B^2_3C^3D_2^3\\
    &\quad+ A_1B^3_3C^3D_3^1 + A_2B^3_3C^3D_3^2 + A_3B^3_3C^3D_3^3\\
\end{align*}
\end{itemize}
\end{proof}

\cleardoublepage
\begin{proof}{\textbf{Exercise 1.10}}\\
\begin{itemize}
\item [1.] To fix $\bm{B}\cdot\bm{C} = B^iC_j$ we need to replace the index $j$
by $i$ because of the definition we have in index notation for the dot product
i.e. should be $\bm{B}\cdot\bm{C} = B^iC_i$.
\item [2.] In the case of $A^{ij} = B^{ik}C_k^?$ since on the LHS we have two
free indexes then on the RHS we should have the same then must be that
$$A^{ij} = B^{ik}C_k^j$$
\item [3.] The expression $N_i = R^k_iD_k^pC_?$ is fixed putting $p$ where the
question mark is i.e.
$$N_i = R^k_iD_k^pC_p$$
Since the only free index in the LHS is $i$.
\item [4.] The expression $L^i_kD^k = R^n_mN^?_nP^?$ is fixed as follows
$$L^i_kD^k = R^n_mN^i_nP^m$$
or as 
$$L^i_kD^k = R^n_mN^m_nP^i$$
Since the only free index in the LHS is $i$.
\item [5.] The expression $\delta^{ij} A_jB_i^k = A^?B_i^k = C^?$ is fixed as
follows
$$\delta^{ij} A_jB_i^k = A^iB_i^k = C^k$$
Since we are rasing the index in the first equality and we are leaving only $k$
as a free index in the last equality.
\item [6.] The expression $N_i = R^k_i\delta^p_kC_? = R_?^?C^?$ is fixed as
follows
\begin{align*}
    N_i &= R^k_i\delta^p_kC_p = R_i^kC^k
\end{align*}
Since we are raising the index of $C_?$ in the last equality.
\end{itemize}
\end{proof}

\cleardoublepage
\begin{proof}{\textbf{Exercise 1.12}}\\
\begin{itemize}
\item[1.]\begin{align*}
    \bm{A}\cdot (\vec{\nabla} f) = A^i \partial_i f
\end{align*}

\item[2.]\begin{align*}
    \vec{\nabla} \cdot \vec{\nabla} f = \partial^i\partial_i f
\end{align*}

\item[3.]\begin{align*}
    \bm{A}(\vec{\nabla} \cdot \bm{B}) = A^i\partial_j B^j \ei{i}
\end{align*}

\item[4.]\begin{align*}
    \nabla^2 f - \frac{1}{v^2}\pdv[2]{f}{t} =
    \partial_i\partial^i f - \frac{1}{v^2}\partial_t^2 f = 0
\end{align*}
\end{itemize}
\end{proof}

\cleardoublepage
\begin{proof}{\textbf{Exercise 1.13}}\\
\begin{itemize}
\item[1.]
Using the product rule we see that
\begin{align*}
    \partial_k(a_{ij}A^iA^j) &= a_{ij}\partial_k(A^iA^j)\\
    &= a_{ij}(A^j(\partial_kA^i) + A^i(\partial_kA^j))\\
    &= 2a_{ij}A^i(\partial_kA^j)
\end{align*}
In the third steps we changed the indices because the summation isn't altered
by swapping the indices.
\item[2.]
We know that the notation $\partial_i x_j$ implies
$\partial_i x_j = \partial x_j/\partial x_i$ so if $j = i$ then
$\partial x_j/\partial x_i = 1$ and if $j \neq i$ we get that
$\partial x_j/\partial x_i = 0$ so we can write that
$\partial_i x_j = \delta_{ij}$.
\\
In the case of $\partial_i x^j$ we must note first that $x^j = \delta^{kj}x_k$
then
\begin{align*}
    \partial_i x^j = \partial_i \delta^{kj}x_k
    = \delta^{kj} \partial_i x_k
    = \delta^{kj} \delta_{ik} = \delta_{i}^j
\end{align*}
Therefore if $i = j$ then $\partial_i x^j = 1$ and if $i \neq j$ we get that
$\partial_i x^j = 0$.
\item [3.] We see that
\begin{align*}
    \nabla \cdot \bm{r} = \partial_i x^i
    = \pdv{x}{x} + \pdv{y}{y} + \pdv{z}{z} = 3
\end{align*}
\item [4.] Using that $|\bm{r}| = \sqrt{x_i x^i}$, the chain rule and the
product rule we see that
\begin{align*}
    \nabla(\ln|\bm{r}|) &= \partial^j (\ln\sqrt{x_ix^i})\ei{j}\\
    &= \frac{1}{2}\partial^j (\ln(x_ix^i))\ei{j}\\
    &= \frac{1}{2} \bigg(\frac{\partial^j x_i x^i}{x_kx^k}\bigg)\ei{j}\\
    &= \frac{1}{2} \bigg(\frac{x^i\partial^j x_i + x_i\partial^j x^i}{x_kx^k}\bigg)\ei{j}\\
    &= \frac{1}{2} \bigg(\frac{x^i\delta_i^j + x_i\delta^{ji}}{x_kx^k}\bigg)\ei{j}\\
    &= \frac{1}{2} \bigg(\frac{x^j + x^j}{x_kx^k}\bigg)\ei{j}\\
    &= \frac{x^j\ei{j}}{x_kx^k}\\
    &= \frac{\bm{r}}{r^2} 
\end{align*}
\item [5.] Let us note that $\nabla^2 = \nabla \cdot \nabla$, so reusing the
result from part 4 we see that
\begin{align*}
    \nabla^2(\ln|\bm{r}|) &= \nabla \cdot \nabla(\ln|\bm{r}|)\\
    &= \partial_j\frac{x^j}{x_kx^k}\\
    &= x^j\partial_j\frac{1}{x_kx^k} + \frac{1}{x_kx^k}\partial_jx^j\\
    &= -x^j\frac{\partial_j x_kx^k}{(x_kx^k)^2} + \frac{3}{x_kx^k}\\
    &= -x^j\frac{x^k\partial_j x_k + x_k\partial_j x^k}{(x_kx^k)^2}
    + \frac{3}{x_kx^k}\\
    &= -x^j\frac{x^k\delta_{jk} + x_k\delta_j^k}{(x_kx^k)^2}
    + \frac{3}{x_kx^k}\\
    &= -\frac{2x_jx^j}{(x_kx^k)^2} + \frac{3}{x_kx^k}\\
    &= -\frac{2r^2}{(r^2)^2} + \frac{3}{r^2}\\
    &= -\frac{2}{r^2} + \frac{3}{r^2}\\
    &= \frac{1}{r^2}
\end{align*}
Where we used that $\partial_jx^j = 3$ from part 3.
\end{itemize}

\end{proof}

\cleardoublepage
\begin{proof}{\textbf{Exercise 1.14}}\\
\begin{itemize}
\item[1.] Equation 1.99 states that
\begin{align*}
    \bm{C} = \delta^{ni}\varepsilon_{ijk}A^jB^k\ei{n}
\end{align*}
Expanding the equation we get that
\begin{align*}
    \bm{C} &= 
    \delta^{11}\varepsilon_{123}A^2B^3\ei{1}
    + \delta^{11}\varepsilon_{132}A^3B^2\ei{1} +\\
    &\quad+ \delta^{22}\varepsilon_{213}A^1B^3\ei{2}
    + \delta^{22}\varepsilon_{231}A^3B^1\ei{2} + \\
    &\quad+ \delta^{33}\varepsilon_{321}A^2B^1\ei{3}
    + \delta^{33}\varepsilon_{312}A^1B^2\ei{3}\\
    &= \varepsilon_{123}A^2B^3\ei{1}
    - \varepsilon_{123}A^3B^2\ei{1}
    - \varepsilon_{231}A^1B^3\ei{2}
    + \varepsilon_{231}A^3B^1\ei{2}\\
    &\quad- \varepsilon_{312}A^2B^1\ei{3}
    + \varepsilon_{312}A^1B^2\ei{3}\\
    &= \varepsilon_{123}(A^2B^3 - A^3B^2)\ei{1}
    + \varepsilon_{231}(A^3B^1 - A^1B^3)\ei{2}\\
    &\quad+ \varepsilon_{312}(A^1B^2 - A^2B^1)\ei{3}\\
    &= (A^2B^3 - A^3B^2)\ei{1} + (A^3B^1 - A^1B^3)\ei{2} 
    + (A^1B^2 - A^2B^1)\ei{3}
\end{align*}
Where we only wrote the non-zero terms and we used the rules of the Levi-Civita
symbol.
\\
Equations (1.100) and (1.101) are essentially the same calculations.
\item[3.] Equation (1.106) states that
\begin{align*}
    \varepsilon_{ijk}\varepsilon^{mnl}
    = \delta_i^m (\delta_j^n\delta_k^l - \delta_j^l\delta_k^n)
    - \delta_i^n (\delta_j^m\delta_k^l - \delta_j^l\delta_k^m)
    + \delta_i^l (\delta_j^m\delta_k^n - \delta_j^n\delta_k^m)
\end{align*}
Then if $l = k$ we get that
\begin{align*}
    \varepsilon_{ijk}\varepsilon^{mnk}
    &= \delta_i^m (\delta_j^n\delta_k^k - \delta_j^k\delta_k^n)
    - \delta_i^n (\delta_j^m\delta_k^k - \delta_j^k\delta_k^m)
    + \delta_i^k (\delta_j^m\delta_k^n - \delta_j^n\delta_k^m)\\
    &= \delta_i^m (3\delta_j^n - \delta_j^n)
    - \delta_i^n (3\delta_j^m - \delta_j^m)
    + \delta_i^k (\delta_j^m\delta_k^n - \delta_j^n\delta_k^m)\\
    &= 2\delta_i^m \delta_j^n
    - 2\delta_i^n \delta_j^m
    + \delta_i^k \delta_j^m\delta_k^n - \delta_i^k \delta_j^n\delta_k^m\\
    &= 2\delta_i^m \delta_j^n - 2\delta_i^n \delta_j^m
    + \delta_j^m\delta_i^n - \delta_j^n\delta_i^m\\
    &= \delta_i^m \delta_j^n - \delta_i^n \delta_j^m
\end{align*}
If $n = j$ and $l = k$ we get that
\begin{align*}
    \varepsilon_{ijk}\varepsilon^{mjk}
    &= \delta_i^m \delta_j^j - \delta_i^j \delta_j^m
    = 3\delta_i^m - \delta_i^m
    = 2\delta_i^m
\end{align*}
And if $m =i$, $n = j$ and $l = k$
\begin{align*}
    \varepsilon_{ijk}\varepsilon^{ijk}
    &= 2\delta_i^i = 2\cdot 3 = 6
\end{align*}

\cleardoublepage
\item[4.] Our guess for the formula equivalent to (1.106) for the case of the
two-dimensional Levi-Civita symbol is
\begin{align*}
    \varepsilon_{ij}\varepsilon^{mn}
    = \delta_i^m \delta_j^n - \delta_i^n \delta_j^m
\end{align*}
We check this guess by computing the components. 
\begin{align*}
    \varepsilon_{12}\varepsilon^{12} &= 1
    = 1 - 0 = \delta_1^1 \delta_2^2 - \delta_1^2 \delta_2^1\\
    \varepsilon_{21}\varepsilon^{12} &= -1
    = 0 - 1 = \delta_2^1 \delta_1^2 - \delta_2^2 \delta_1^1\\
    \varepsilon_{12}\varepsilon^{21} &= -1
    = 0 - 1 = \delta_1^2 \delta_2^1 - \delta_1^1 \delta_2^2\\
    \varepsilon_{21}\varepsilon^{21} &= 1
    = 1 - 0 = \delta_2^2 \delta_1^1 - \delta_2^1 \delta_1^2
\end{align*}
For the components where $i = j$ we have that
\begin{align*}
    \varepsilon_{ii}\varepsilon^{mn} &= 0
    = \delta_i^m \delta_i^n - \delta_i^n \delta_i^m
\end{align*}
And for the components where $m = n$ we get that
\begin{align*}
    \varepsilon_{ij}\varepsilon^{mm} &= 0
    = \delta_i^m \delta_j^m - \delta_i^m \delta_j^m
\end{align*}
Therefore our guess is correct.
\\
Let us compute the identities similar to (1.107) as follows
\begin{align*}
    \varepsilon_{ij}\varepsilon^{mj}
    &= \delta_i^m \delta_j^j - \delta_i^j \delta_j^m
    = 3\delta_i^m - \delta_i^m
    = 2\delta_i^m
    \\
    \varepsilon_{ij}\varepsilon^{ij}
    &= 2\delta_i^i = 6
\end{align*}


\end{itemize}
\end{proof}

\cleardoublepage
\begin{proof}{\textbf{Exercise 1.15}}\\
Let $A^{ij}$ be any matrix then we can write it as
\begin{align*}
    A^{ij} = \frac{1}{2}(A^{ij} + A^{ji}) + \frac{1}{2}(A^{ij} - A^{ji})
\end{align*}
Let us consider the two terms as independent matrices.
\\
If we define $B^{ij} =\frac{1}{2}(A^{ij} + A^{ji})$ we see that 
\begin{align*}
    B^{ji} = \frac{1}{2}(A^{ji} + A^{ij}) = \frac{1}{2}(A^{ij} + A^{ji}) = B^{ij}
\end{align*}
So we see that $B^{ij}$ is a totally symmetric matrix.
\\
On the other hand, let us define $C^{ij} =\frac{1}{2}(A^{ij} - A^{ji})$, then
we see that 
\begin{align*}
    C^{ji} = \frac{1}{2}(A^{ji} - A^{ij}) = -\frac{1}{2}(A^{ij} - A^{ji}) = -C^{ij}
\end{align*}
So $C^{ij}$ is a totally antisymmetric matrix.
\\
Therfore we may write $A^{ij}$ as 
\begin{align*}
    A^{ij} = A^{(ij)} + A^{[ij]}
\end{align*}
Where $B^{ij} = A^{(ij)}$ and $C^{ij} = A^{[ij]}$ and hence
\begin{align*}
    A^{(ij)} = \frac{1}{2}(A^{ij} + A^{ji})\qquad
    A^{[ij]} = \frac{1}{2}(A^{ij} - A^{ji})
\end{align*}
\end{proof}

\cleardoublepage
\begin{proof}{\textbf{Exercise 1.16}}\\
To decompose the matrix
\begin{align*}
    A = \begin{bmatrix}
        1 & 1 & 2\\
        -3 & 2 & -3\\
        4 & 3 & 3
    \end{bmatrix}
\end{align*}
We need to compute each element of the totally symmetric matrix $A_s$ and
the totally antisymmetric matrix $A_a$, for this we use the equations (1.113)
as follows
\begin{align*}
    A^{(11)}_s &= \frac{1}{2}(A^{11}_s + A^{11}_s) = \frac{1}{2}(1 + 1) = 1\\
    A^{(12)}_s &= \frac{1}{2}(A^{12}_s + A^{21}_s) = \frac{1}{2}(1 - 3) = -1\\
    A^{(13)}_s &= \frac{1}{2}(A^{13}_s + A^{31}_s) = \frac{1}{2}(2 + 4) = 3
\end{align*}
Given that $A_s$ is totally symmetric we know that $A^{(21)}_s = A^{(12)}_s$
and $A^{(31)}_s = A^{(13)}_s$ hence the only element left to compute is
$A^{(23)}_s = A^{(32)}_s$
\begin{align*}
    A^{(23)}_s &= \frac{1}{2}(A^{23}_s + A^{32}_s) = \frac{1}{2}(-3 + 3) = 0
\end{align*}
Then the totally symmetric matrix is 
\begin{align*}
    A_{s} = \begin{bmatrix}
        1 & -1 & 3\\
        -1 & 2 & 0\\
        3 & 0 & 3
    \end{bmatrix}
\end{align*}
For the totally antisymmetric matrix $A_a$ we compute what follows
\begin{align*}
    A^{[11]}_a &= \frac{1}{2}(A^{11}_a - A^{11}_a) = \frac{1}{2}(1 - 1) = 0\\
    A^{[12]}_a &= \frac{1}{2}(A^{12}_a - A^{21}_a) = \frac{1}{2}(1 + 3) = 2\\
    A^{[13]}_a &= \frac{1}{2}(A^{13}_a - A^{31}_a) = \frac{1}{2}(2 - 4) = -1
\end{align*}
And computing the element $A^{[23]}_a$ we have all we need to get $A_a$
\begin{align*}
    A^{[23]}_a &= \frac{1}{2}(A^{23}_a - A^{32}_a) = \frac{1}{2}(-3 - 3) = -3
\end{align*}
Then the matrix $A_a$ is
\begin{align*}
    A_{a} = \begin{bmatrix}
        0 & 2 & -1\\
        -2 & 0 & -3\\
        1 & 3 & 0
    \end{bmatrix}
\end{align*}
Finally, we can compute the sum $A_s + A_a$ to check that the matrices are
correct, we should get $A$.
\begin{align*}
    A_{s} + A_{a} = \begin{bmatrix}
        1 & -1 & 3\\
        -1 & 2 & 0\\
        3 & 0 & 3
    \end{bmatrix} + \begin{bmatrix}
        0 & 2 & -1\\
        -2 & 0 & -3\\
        1 & 3 & 0
    \end{bmatrix}
    = \begin{bmatrix}
        1 & 1 & 2\\
        -3 & 2 & -3\\
        4 & 3 & 3
    \end{bmatrix}
    = A
\end{align*}
\end{proof}

\cleardoublepage
\begin{proof}{\textbf{Exercise 1.17}}\\
\begin{itemize}
\item [1.] Let $\varepsilon_{ijk} A^k = B_{ij}$ then multiplying both sides by
$\varepsilon^{ijl}$ we get that
\begin{align*}
    \varepsilon_{ijk}\varepsilon^{ijl} A^k &= \varepsilon^{ijl}B_{ij}\\
    2\delta_{k}^l A^k &= \varepsilon^{ijl}B_{ij}\\
    2 A^l &= \varepsilon^{ijl}B_{ij}\\
    A^l &= \frac{1}{2}\varepsilon^{ijl}B_{ij}
\end{align*}
\item [2.] Let now $A^{jk}$ be totally antisymmetric and let also 
$\varepsilon_{ijk} A^{jk} = B_{i}$ then
\begin{align*}
    \varepsilon^{inm}\varepsilon_{ijk} A^{jk} &= \varepsilon^{inm}B_{i}\\
    (\delta_j^n\delta_k^m - \delta_k^n\delta_j^m) A^{jk} &= \varepsilon^{inm}B_{i}\\
    \delta_j^n\delta_k^mA^{jk} - \delta_k^n\delta_j^m A^{jk} &= \varepsilon^{inm}B_{i}\\
    A^{nm} - A^{mn} &= \varepsilon^{inm}B_{i}
\end{align*}
But $A$ is totally antisymmetric so $A^{mn} = -A^{nm}$ and hence
\begin{align*}
    A^{nm} + A^{nm} &= \varepsilon^{inm}B_{i}\\
    A^{nm} &= \frac{1}{2}\varepsilon^{inm}B_{i}
\end{align*}
\item [3.] Let now $A^{jk}$ be totally symmetric and let also 
$\varepsilon_{ijk} A^{jk} = B_{i}$ then
\begin{align*}
    \varepsilon^{inm}\varepsilon_{ijk} A^{jk} &= \varepsilon^{inm}B_{i}\\
    \varepsilon^{nmi}\varepsilon_{jki} A^{jk} &= \varepsilon^{inm}B_{i}\\
    (\delta_j^n\delta_k^m - \delta_j^m\delta_k^n) A^{jk} &= \varepsilon^{inm}B_{i}\\
    \delta_j^n\delta_k^mA^{jk} - \delta_j^m\delta_k^n A^{jk} &= \varepsilon^{inm}B_{i}\\
    A^{nm} - A^{mn} &= \varepsilon^{inm}B_{i}\\
    A^{nm} - A^{nm} &= \varepsilon^{inm}B_{i}\\
    0 &= B_i
\end{align*}
Where we used that $A$ is totally symmetric so $A^{mn} = A^{nm}$, then since 
$B_i = 0$ we have that $\varepsilon_{ijk}A^{jk} = 0$.
\item [4.] If $A^{jk}$ is neither totally symmetric nor totally antisymmetric
then we get that
\begin{align*}
    A^{nm} - A^{mn} &= \varepsilon^{inm}B_{i}
\end{align*}
But we can write $A^{nm}$ as the sum of a totally symmetric and a totally
antisymmetric matrix so
\begin{align*}
    A^{(nm)} + A^{[nm]} - A^{(mn)} - A^{[mn]} &= \varepsilon^{inm}B_{i}\\
    A^{(nm)} + A^{[nm]} - A^{(nm)} + A^{[nm]} &= \varepsilon^{inm}B_{i}\\
    2A^{[nm]} &= \varepsilon^{inm}B_{i}\\
    A^{[nm]} &= \frac{1}{2}\varepsilon^{inm}B_{i}
\end{align*}
So as we saw in part 2 the antisymmetric matrix will be of the form
$\frac{1}{2}\varepsilon^{inm}B_{i}$ but we cannot say anything about the
symmetric part.

\item [5.] Let $\delta^{ik}\varepsilon_{ijn}A^n = B_j^k$ then
\begin{align*}
    \delta^{ik}\varepsilon_{ijn}A^n &= B_j^k\\
    {\varepsilon^k}_{jn} A^n &= B_j^k\\
    {\varepsilon_k}^{jl} {\varepsilon^k}_{jn} A^n &= {\varepsilon_k}^{jl} B_j^k\\
    2\delta^l_nA^n &= {\varepsilon_k}^{jl}B_j^k\\
    A^l &= \frac{1}{2}{\varepsilon_k}^{jl}B_j^k
\end{align*}
\end{itemize}
\end{proof}
\end{document}